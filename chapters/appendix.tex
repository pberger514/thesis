\appendix

\chapter{Appendices to Chapter \ref{chap:signal}}

Peak Patch stuff.

\chapter{Appendices to Chapter \ref{chap:hol}}

\section{Regridding Algorithm}
\label{app:hol:sec:regrid}

\textit{This section is based on a document provided by Richard Shaw, who also wrote the regridding code for the CHIME pipeline.}

The CHIME timestream is a set of semi-regular samples $\tilde{v}$ from an underlying sky
signal. As the sky signal is band-limited we can uniquely define the sky by a set of
regularly spaced samples, provided the sampling rate exceeds the Nyquist rate. We'll denote those sky samples by $v$, and they are located at times $t_n = n T$.

For this band-limited signal we can relate the value of the timestream at any time $t$, to the regularly-spaced grid of samples
\begin{equation}
\tilde{v} = \sum_n \mathrm{sinc}\left(\frac{t - t_n}{T}\right)v(t_n).
\end{equation}
However, the interpolation kernel is of infinite length which is problematic computationally,
so we would like something that is much more local. A common approximation
is to use the Lanczos interpolation kernel instead, which is essentially a trucated $\mathrm{sinc}$,
\begin{align}
R(x) = \begin{cases}
\mathrm{sinc}(x)\mathrm{sinc}(x/a) & x<a \\
0 & x \geq a
\end{cases}
\end{align}
and can be used to interpolate a timestream like
\begin{align}
\tilde{v}(t) = \sum_n R\left( \frac{t - t_n}{T}\right) v(t_n).
\end{align}
This kernel is width 2$a$, and so at most couples 4$a$+1 samples of $v$.

The measurement process draws a noisy sample from the underlying signal. This
can be written as 
\begin{align}
\tilde{\mathbf{v}} = \mathbf{R}\mathbf{v} + \mathbf{n},
\end{align}
where $\mathbf{R}$ is the interpolation matrix and $\mathbf{N}$ is the additive noise. In our case there are essentially two sources of noise, the thermal noise of the instrument and noise from
Radio Frequency Interference (RFI), that we denote as $\mathbf{n}_r$. Each of these is described
by a noise covariance matrix. The thermal noise, which will generally be uncorrelated with a smoothy changing variance is denoted as $\mathbf{N}_t$. Since the RFI is much brighter than the signal, the noise at the locations of flagged RFI is essentailly infinite, and zero elsewhere. Therefore the inverse of the combined noise covariance, $\mathbf{N} = \mathbf{N}_t + \mathbf{N}_r$, as a function of time will have a smoothly varying value along the diagonal except at locations of flagged RFI where it is zero.

We estimate the regularly gridded timestream by computing
\begin{align}
\hat{\mathbf{v}} = \mathbf{CR}^T \mathbf{N}^{-1} \tilde{\mathbf{v}},
\end{align}
where
\begin{align}
\mathbf{C} = (\mathbf{S} + \mathbf{R}^T \mathbf{N}^{-1} \mathbf{R})^{-1},
\end{align}
is the data covariance matrix. This can be efficiently evaluate under the following assumptions:
\begin{itemize}
\item The noise covariance $\mathbf{N}$ is diagonal.
\item The signal covariance $\mathbf{S}$ is band diagonal and constant across frequencies and baseline pairs (although the latter assumption is not strictly necessary it is seen to peform well).
\item The sampling kernel has compact support, such that there is a finite coupling
length between points in the signal timestream. The Lanczos kernel discussed
above has this property. This ensures that the quantityis band diagonal (with width 2 elements above and below the diagonal).
\end{itemize}

\section{Holography-photometry correspondence}
\label{app:hol:sec:photometry}

Flux calibrating holography measurements requires holographic visibilities of both the source in question and the calibrator, given by\begin{subequations}
\begin{align}
V^{\rm src}_{i,{\rm 26}} &= g^{\rm src}_i g^{*{\rm src}}_{\rm 26} A_i(\delta_{\rm src}) A^*_{\rm 26} T_{\rm src} ~~ \left[ {\rm corr~units} \right] \label{holvis1}
\\ 
V^{\rm cal}_{i,{\rm 26}} &= g^{\rm cal}_i g^{*{\rm src}}_{\rm 26} A_i(\delta_{\rm cal}) A^*_{\rm 26} T_{\rm cal} ~~  \left[ {\rm corr~units} \right], \label{holvis2}
\end{align}
\end{subequations}
on boresight. Eqs. \ref{holvis1} and \ref{holvis2} assume that the 26 m's beam is constant with declination, while any of the CHIME antenna beams, indexed by $i=1,\dots,N_{\rm feed}$, may vary. At this stage the units are correlator units. We then divide out the point source gain for the calibrator CHIME transit nearest the holography observation 
\begin{align}
g^{\rm cal}_i A_i(\delta_{\rm cal}) T^{1/2}_{\rm cal} ~~  \left[ \frac{\rm corr~units}{\rm Jy} \right]^{1/2}.
\end{align}
Each individual antenna's gain carries only a factor of the calibrator flux to the $1/2$, and we correct only for the CHIME gain in this way.
\begin{subequations}
\begin{align}
\overline{V}^{\rm src}_{i,{\rm 26}} &= g^{*{\rm src}}_{\rm 26}A^*_{\rm 26}~\frac{A_i(\delta_{\rm src})}{A_i(\delta_{\rm cal})}  \frac{T_{\rm src}}{T_{\rm cal}^{1/2}} ~~ \left[ {\rm corr~units \cdot Jy} \right]^{1/2} \label{cholvis1}
\\ 
\overline{V}^{\rm cal}_{i,{\rm 26}} &=  g^{*{\rm src}}_{\rm 26} A^*_{\rm 26}~\frac{\cancel{A_i(\delta_{\rm cal})}}{\cancel{A_i(\delta_{\rm cal})}} T_{\rm cal}^{1/2} ~~  \left[ {\rm corr~units \cdot Jy} \right]^{1/2},  \label{cholvis2}
\end{align}
\end{subequations}
where in Eq. \ref{cholvis2} we see that the dependence of the CHIME beam at transit cancels. The calibrated visibilities $\overline{V}$ now have awkward units, and the spectrum is multiplied by the 26m band pass. However, taking the ratio of Eqs. \ref{cholvis1} and \ref{cholvis2}
\begin{align}
\frac{\overline{V}^{\rm src}_{i,{\rm 26}}}{\overline{V}^{\rm cal}_{i,{\rm 26}}} = \frac{A_i(\delta_{\rm src})}{A_i(\delta_{\rm cal})} \frac{T_{\rm src}}{T_{\rm cal}},
\end{align}
a dimensionless quantity. This just means that we have normalized the beams to have amplitude 1 and phase 0 at the peak location of the calibrator. Next, we divide out ratio of the expected source and calibrator fluxes
\begin{align}
\frac{\overline{V}^{\rm src}_{i,{\rm 26}}}{\overline{V}^{\rm cal}_{i,{\rm 26}}} \frac{T_{\rm cal}}{T_{\rm src}} = \frac{A_i(\delta_{\rm src})}{A_i(\delta_{\rm cal})}\equiv \hat{A}_i(\delta_{\rm src} | \delta_{\rm cal}).
\end{align}
This step is quite important, since from it we obtain $\hat{A}_i(\delta_{\rm src} | \delta_{\rm cal})$, our estimate of the beam at the location of the source given the calibrator. In principle we can stop here, since we can now convert these beams to beam transfer matrices. However we would like to validate the beams by comparing to photometry from the Pathfinder, that has many fewer systematics than the holography. 

To create ring maps we first bin the baselines by their North-South separation. Considering only the NS direction the calibrated visibilities are then,
\begin{align}
V_{(i-j)} = \int {\rm d} \delta \left\langle \frac{A_i(\delta)}{A_i(\delta_{\rm cal})} \frac{A_j(\delta)}{A_j(\delta_{\rm cal})}^* \right\rangle_{(i-j)} T(\delta) ~ {\rm e}^{i2\pi \frac{(i-j) \Delta x}{\lambda}(\sin{\delta} - \sin{\delta_{\rm cal}})}, 
\end{align}
where $\Delta x = 0.3~{\rm m}$ for CHIME. The visibilities are phased up to the point source, meaning the geometric phase factor is fixed to be zero there. If there's a point source in the beam the sky temperature is just $T(\delta) = \delta_{\rm D}(\delta - \delta_{\rm src}) T_{\rm src}$, and we can compute the integral
\begin{align}
V_{(i-j)} = \left\langle \frac{A_i(\delta_{\rm src})}{A_i(\delta_{\rm cal})} \frac{A_j(\delta_{\rm src})}{A_j(\delta_{\rm cal})}^* \right\rangle_{(i-j)} T_{\rm src} ~ {\rm e}^{ix(\theta_{\rm src} - \theta_{\rm cal})},
\end{align}
where we have defined $x \equiv i2\pi \frac{(i-j) \Delta x}{\lambda}$ and $\theta \equiv \sin{\delta}$. To do aperture photometry we then compute the inverse Fourier transform of this quantity,
\begin{align}
\widetilde{V_{(i-j)}}(\theta') \propto \int {\rm d} x \left\langle \frac{A_i(\delta_{\rm src})}{A_i(\delta_{\rm cal})} \frac{A_j(\delta_{\rm src})}{A_j(\delta_{\rm cal})}^* \right\rangle_{(i-j)} T_{\rm src} ~ {\rm e}^{ix(\theta_{\rm src} - \theta_{\rm cal})} {\rm e}^{-ix\theta'},
\end{align}
and for simplicity we then evaluate it at the location of the source
\begin{align}
\widetilde{V_{(i-j)}}(\theta_{\rm src}) \propto \sum_{(i-j)} \left\langle \frac{A_i(\delta_{\rm src})}{A_i(\delta_{\rm cal})} \frac{A_j(\delta_{\rm src})}{A_j(\delta_{\rm cal})}^* \right\rangle_{(i-j)} T_{\rm src}. \label{photometry}
\end{align}
We see that if $\theta_{\rm cal} = 0$, the phase factors simply cancel and we obtain the average of all the visibilites (after discretizing the Fourier the transform).

If only intercylinder baselines have been used to create the ring map, after diving by the expected flux of the source this can then be compared to the average cross-cylinder beam obtained from holography
\begin{equation}
\frac{1}{N_{\rm intercyl}}\sum_{{\rm cyl}_i \ne{\rm cyl}_j}  A_i(\delta_{\rm src} | \delta_{\rm cal})A_j(\delta_{\rm src} | \delta_{\rm cal})^*
\end{equation}

\section{A Model of Polarization}
\label{app:hol:sec:polmodel}

To be able to produce beam transfer matrices from intensity-only estimates of the beam, we require a model for the polarized response of the telescope. The beam transfer functions are given by
\begin{align}
B^X_{ij}(\hat{n}; \phi) = \frac{2}{\Omega_{ij}} A^a_i(\hat{n}) A^b_j(\hat{n}) P_{ab}^X e^{2\pi i \hat{n}\cdot\vec{u}_{ij}(\phi)},
\end{align}
where $X$ labels the Stokes parameters, $X=\{I, Q, U, V\}$, and $P_{ab}^X$ are the Pauli matrices. That is $P_{ab}^X$ projects from the telescope's polarization into the Stokes parameters on the sky. If we assume that the polarization response for each feed is the same then the beam amplitude functions are separable,
\begin{align}
A^a_i(\hat{n}) = A_i(\hat{n})p^a(\hat{n}).
\end{align}
We further assume that the polarization pattern is given by a dipole in the direction $\hat{d}$,
\begin{align}
\hat{p}(\hat{n}, \hat{d}) =\frac{\hat{d} - (\hat{n}\cdot\hat{d}) \hat{n}}{(1 - (\hat{n}\cdot\hat{d})^2)^{1/2}}.
\end{align}
This can then be applied either to the averaged beam amplitude functions $A_i$  or its product can be applied to single or averaged baselines $A_jA_j$.

\section{A ray tracing model for the CHIME beam}
\label{app:hol:sec:model}

\begin{figure}[h!]
\begin{center}
\includegraphics[scale=.4]{figures/hol_other/cylpic.pdf}
\caption{The relevant geometry and definitions for the problem of a two dimensional plane wave encountering a parabolic aperture.}
\end{center}
\label{cylpic}
\end{figure}

Consider a plane electromagnetic wave traveling through free space striking a cylindrical reflector. If the source of the plane wave is directly over head, that is if the angle between the source and the axis of symmetry of the parabola $\theta_i$ is zero, then each wave front reaches the feed in phase after reflecting off the cylinder. Here $i=1,\dots,N$, where $N$ is number angles at which the response to the point source has been sampled. However, invoking the Huygens' Principle, for an angle $\theta_i\neq 0 $, the section of the wave front striking the cylinder at any location $(x_j, y_j)$ will contribute to the signal at the feed. Furthermore it travels a distance $\scripty{r}_{ij}$ which maybe be different from the other sections, since $\vec{\scripty{r}_{ij}} = \vec{R}(\theta_i) + \vec{r}(x_j)$ for $\vec{R}(\theta_i)$ the vector to the source and $\vec{r}(x_j)$ the vector from the focus to the location on the cylinder. The goal, therefore, is to solve for the complex contribution $g_j$ to the response at the feed, from each section of cylinder. The geometry of the parabola constrains
\begin{equation}
y_j = \frac{x_j^2}{4F},
\end{equation}
where $F$ is the focal length. We discretize the cylinder in the same manner as the angles $\theta_i$. We centre our coordinate system at the feed so that $-x_0 = x_N,~-x_1=x_{N-1},\dots$ Refer to Figure \ref{cylpic} for the relevant geometry and definitions.

Traveling through free space, a plane wave acquires a phase $e^{ik\mathcal{R}}$, where $k = 2\pi/\lambda$, $\lambda$ is the wavelength, and $\mathcal{R}$ is the total distance of travel. The distance our discrete wave fronts travel from the source is just $\mathcal{R}_{ij} = \scripty{r}_{ij}  + r_j$, before arriving at the feed. Therefore the response at the feed is just the sum of the contributions from each section of cylinder
\begin{equation}
E(\theta_i) = E_0 \sum_j e^{ik(\scripty{r}_{ij}  + r_j)} g_j.
\label{sumovercyl}
\end{equation}
We decompose our vectors in Cartesian coordinates as
\begin{align}
\vec{R}_i &= \left(R \sin(\theta_i), R \cos(\theta_i)\right) \\
\vec{r}_j &= \left(x_j, -(F-\frac{x_j^2}{4F})\right) 
\end{align}
Therefore
\begin{align}
\vec{\scripty{r}_{ij}} &= \left(R\sin(\theta_i) + x_j, R \cos(\theta_i) + y_j)\right) \\
\rightarrow |\vec{\scripty{r}_{ij}}| &= R \left( (\sin(\theta_i) + x_j/R)^2 + (\cos(\theta_i) + y_j/R )^2 \right)^{1/2}
\end{align}
We then take the far field limit, expanding in Taylor series and keeping only up to terms linear in $x/R$ and $y/R$ to obtain
\begin{equation}
|\vec{\scripty{r}_{ij}}| = R + x_j\sin(\theta_i) + y_j\cos(\theta_i) 
\label{taylor}
\end{equation}
However, in interferometry we can only hope to measure relative phases. In the case of holography the tracking dish serves as the reference beam and all phases are measured relative to its. We therefore ask for the differential phase acquired relative to the zero of our coordinate system, by subtracting the distance to the source from (\ref{taylor}). Finally, (\ref{sumovercyl}) becomes
\begin{equation}
E(\theta_i) = E_0 \sum_j e^{ik( x_j\sin(\theta_i) - (F_i - \frac{x_j^2}{4F})\cos(\theta_i)   + r_j)} g_j.
\label{final1d}
\end{equation}
To project on to the cylinder we simply invert (\ref{final1d}) to solve for $g_j$. For a perfect cylinder, and if the feed itself has no intrinsic phase which depends on the angle to $(x_j, y_j)$, then the $g_j$ are real and correspond to the aperture illumination at that point. In this model, it's then easy to see that some distortion in the cylinder produces a differential phase which is not removed by inverting (\ref{final1d}). The phase of the resulting complex $g_j$ corresponds to the amplitude of the distortion in units of wavelength.


\section{Noise in interferometers}
\label{app:hol:sec:noise}

\subsection{Simple baseline-based calculation} 
\label{app:hol:sec:noise:ss:basecalc}

Here we present an analytic analysis of the noise in interferometers in the presence of cross talk. To gain some intution into the kinds of operations we are dealing with we will first perform the standard calculation with no cross talk, which is heavily based on Ref. \citep{kulkarni}. We begin with a simple model for the voltages obtained at the correlator, which assumes that the received electric field is a just a scalar, and that we observe a simple sum of this signal and thermal noise fluctuations. After the Fourier transform stage of the correlator we have
\begin{equation}
\overline{E}_i(\nu) = E_i(\nu)+ n_i(\nu),
\end{equation}
where overbars denote observed quantities. In this section (\ref{app:hol:sec:noise:ss:basecalc}) and the following (\ref{app:hol:sec:noise:ss:basecalcx}), we will neglect any correlations between frequencies or times and so will drop any explicit reference. We are therefore interested only in how the various baselines interact at any one frequency or time. In Section \ref{app:hol:sec:noise:ss:fullcalc} and \ref{app:hol:sec:noise:ss:fullcalcx} we will notice that even the signal component of the data is made up of delayed signals with path differences set by the baselines.
\subsubsection{Signal}
First, we can construct the two point function of $\overline{E}_i$, the observed visibilities
\begin{align}
\overline{V}_{ij} \equiv \left\langle \overline{E}_i \overline{E}_j^* \right\rangle &= \left\langle (E_i + n_i)(E_j + n_j)^* \right\rangle = \left\langle E_iE_j^* + E_in_j^* + n_iE_j^* + n_in_j^*\right\rangle \nonumber \\ 
&= \left\langle E_iE_j^* \right\rangle + \cancel{\left\langle E_in_j^* \right\rangle} + \cancel{\left\langle n_iE_j^*\right\rangle} +\left\langle n_in_j^* \right\rangle
\nonumber \\ 
&= V_{ij} + \delta_{ij} N, \label{obsvis}
\end{align}
where N is the noise equivalent flux density (NEFD), related to the system temperature through $N=\frac{1}{2}kT_{\rm sys} / \eta A$. This is the well-known result that diagonal components of the observed visibility triangle, the auto-correlations, receive a contribution to the signal from the system temperature. Conversely for the off-diagonals, the cross-correlations, the noise fluctuations at various antennas do not correlate, and the received signal level is therefore unbiased.
\subsubsection{Covariance}
Next, we would like to understand how the signal and noise contribute to the fluctuation power in the visibilities themselves, keeping in mind that the visibilities are actually a two-point function of the voltages. To do so, we should compute the full covariance of the visibilities, which for the fringe amplitude is given by
\begin{equation}
{\rm Cov}(\overline{V}_{ij}, \overline{V}_{i'j'}) \equiv \left\langle \overline{E}_i \overline{E}_j^*(\overline{E}_{i'} \overline{E}_{j'}^*)^* \right\rangle  -  \left\langle \overline{E}_i \overline{E}_j^* \right\rangle \left\langle \overline{E}_{i'} \overline{E}_{j'}^* \right\rangle^*  = \left\langle \overline{E}_i \overline{E}_j^* \overline{E}_{i'}^* \overline{E}_{j'} \right\rangle - \left\langle \overline{E}_i \overline{E}_j^* \right\rangle \left\langle \overline{E}_{i'}^* \overline{E}_{j'} \right\rangle. \label{fullcov}
\end{equation} 
To gain some more information we can then consider only its diagonal ($i=i', j=j'$) component, the variance of that visibility. Performing the Wick expansion of Eq. \ref{fullcov} greatly simplifies the expansion,
\begin{align}
\left\langle \overline{E}_i \overline{E}_j^* \overline{E}_{i'}^* \overline{E}_{j'} \right\rangle 
&= \left\langle \overline{E}_i \overline{E}_j^* \right\rangle\left\langle \overline{E}_{i'}^* \overline{E}_{j'} \right\rangle  
+ \left\langle \overline{E}_i \overline{E}_{i'}^* \right\rangle\left\langle \overline{E}_{j}^* \overline{E}_{j'} \right\rangle  
+ \cancel{\left\langle \overline{E}_i \overline{E}_{j'} \right\rangle\left\langle\overline{E}_{j}^* \overline{E}_{i'}^* \right\rangle}  - \left\langle \overline{E}_i \overline{E}_j^* \right\rangle \left\langle \overline{E}_{i'}^* \overline{E}_{j'} \right\rangle.
\nonumber \\ & = \left\langle \overline{E}_i \overline{E}_{i'}^* \right\rangle\left\langle \overline{E}_{j}^* \overline{E}_{j'} \right\rangle  \label{wickexpansion}
\end{align}
A simple way of understanding why the second to last term on the first line should be zero is to consider a pure Fourier mode $\cos{\frac{2\pi}{T} t} + i \sin{\frac{2\pi}{T} t}$ and average over its period. Eq. \ref{wickexpansion} can then be expressed in terms of products of observed visibilities, but note the permutation of the indices
\begin{align}
{\rm Cov}(\overline{V}_{ij}, \overline{V}_{i'j'}) &= \overline{V}_{ii'}\overline{V}_{jj'}^*  \label{fullcov-obsvis}
\\ & = (V_{ii'} + \delta_{ii'} N)(V_{jj'} + \delta_{jj'} N)^*
\end{align}
To understand this expression further, we can inspect various cases for the elements of the covariance. For example, we take the diagonal component $(i=i', j=j')$ component to obtain the variance of each baseline.
\begin{align}
\sigma^2_{ij} = (S + N)_{i}(S + N)^*_{j},
\end{align}
where $S$ is now the flux of the source. We have obtained the (again well known) result that variance in the visibilities stems from the square of the antenna and system temperatures at the feeds in question. Futher, for the non-diagonal elements, there are two more cases of interest (which exhaust the possible shapes in the covariance): the $(i=i', j\neq j')$ case, where a single feed is shared between the two cross correlations we are considering, and the $(i\neq i', j\neq j')$, where no feeds are shared. In these cases we have
\begin{align}
(i=i', j\neq j') &\rightarrow (S + N)_{i} V_{jj'},
\\{\rm and~}(i\neq i', j\neq j') &\rightarrow V_{ii'} V^{*}_{jj'}.
\end{align}
The take home message here is that for the non-diagonal elements, the covariance drops since the noise no longer correlates, and also by the cross-correlation coefficient, that depends on the correlation properties of the source in question and its signal-to-noise.

\subsection{Cross-talk talk between baselines} 
\label{app:hol:sec:noise:ss:basecalcx}

We would now like to gain some intuition into the kinds of terms that cross-talk can introduce. After the Fourier transform we have,
\begin{equation}
\overline{E}_i = E_i + n_i + \sum_{k} \alpha_{ik} (E_i + n_i) {\rm e}^{2\pi i \tau_{ik}}. \label{pre-xtalkmodel}
\end{equation}
We have extended the model to include attenuated and delayed versions of the voltages at each other antenna, which depend on the $N_{\rm feed}(N_{\rm feed} + 1)/2$ on attenuation parameters $\alpha_{ik}$ and the $N_{\rm feed}(N_{\rm feed} + 1)/2$ delays $\tau_{ik}$. This seems like a reasonable model -- assuming that the digitized noisy voltages you would have received without cross-talk become delayed and attenuated. However consider the analog chain, there's no reason thermal noise inside some cable can't combine with the signal to reflect off a mismatched impedance and be rebroadcasted before reaching the correlator. This logic would lead to quite a complicated model, though, with delays and parameters for each component that could be contributing noise. For now let's consider a model where we allow different attenuation parameters for the rebroadcasted signal and noise one for each antenna's complete analog chain,
\begin{align}
\overline{E}_i = E_i + n_i + \sum_{k} \alpha_{ik} E_i {\rm e}^{2\pi i \tau_{\alpha,ik}} + \sum_{k} \beta_{ik} n_i {\rm e}^{2\pi i \tau_{\beta,ik}}. \label{xtalkmodel}
\end{align}
In practice, any response of the interferometer to actual sky signal is measured by holography. Furthermore, one expects the average of the signal over the entire sky (as well as the first few modes) to be small. Bright sources such as the galaxy or point sources only occupy small regions, and so their flux should average down over large, mostly dark, sky areas. This is in contrast to the extremely bright striping features that we observe in ring maps, which do not average down over a day, and oscillate slowly. It's therefore reasonable to assume that the cross-talk coupling parameters measured from the striping in ring maps are in fact the $\beta$ parameters. In the following therefore, while $\tau_\beta$ and $\tau_\alpha$ are in general distinct we will simplify notation, while keeping in mind that it is the $\tau_\beta$'s we are interested in.

\subsubsection{Signal}
We can compute the modification to the visibilities by cross-talk by following the calculation in Eq. \ref{obsvis} but now inserting Eq. \ref{xtalkmodel}. We work to linear order in the attenuation parameters. For simplicity we also define $\omega_{ij} \equiv 2\pi i \tau_{ij}$. We obtain
\begin{align}
\overline{V}_{ij} = &~ V_{ij} + \delta_{ij} N
\nonumber \\
&+ \sum_k \alpha_{ik} V_{kj} {\rm e}^{\omega_{ik}} + \sum_l \alpha^*_{jl} V_{il} {\rm e}^{-\omega_{jl}}
\nonumber \\
&+ \beta_{ij} N e^{\omega_{ji}} + \beta_{ji}^*N e^{- \omega_{ji}}
\label{xtalk-signal} \\ \nonumber
\equiv &~ V_{ij} + \delta_{ij} N
\\ \nonumber &+ \widetilde{V}_{ij}
\\ &+ \widetilde{N}_{ij}. \label{tilde-def}
\end{align}
There are two new contributions to Equation \ref{xtalk-signal}, which we have tentatively denoted with tildes to mean ``delayed versions'' of the visibilities and noise, $\widetilde{V}_{ij}$ and $\widetilde{N}_{ij}$, respectively. The first is the addition of delayed and attenuated versions of the visibilities, which for any pair of feeds in the correlation triangle, is a sum over all of their cross-correlations. Secondly we see that in the presence of cross-talk, even cross-correlations can receive a biasing contribution from the system temperature, which results from a feed's own thermal noise fluctuations being broadcast to the others, for later correlation.

\subsubsection{Covariance}
Referring back to Eq. \ref{fullcov-obsvis}, the full covariance of the visibilities can be expressed as a product of the observed visibilities (but with permuted indices),
\begin{align}
{\rm Cov}(\overline{V}_{ij}, \overline{V}_{i'j'}) =& \overline{V}_{ii'}\overline{V}_{jj'}^* \label{fullcov-obsvis-xtalk}
\\ =& (V_{ii'} + \delta_{ii'}N + \widetilde{V}_{ii'} + \widetilde{N}_{ii'})(V_{jj'} + \delta_{jj'}N + \widetilde{V}_{jj'} + \widetilde{N}_{jj'})^* 
\\ \simeq &(V_{ii'} + \delta_{ii'}N)(V_{jj'} + \delta_{jj'}N)^*  
\nonumber \\ &+ (V_{ii'} + \delta_{ii'}N)(\widetilde{V}_{jj'} + \widetilde{N}_{jj'})^* + (\widetilde{V}_{ii'} + \widetilde{N}_{ii'})(V_{jj'} + \delta_{jj'}N)^*
\label{fullcov-xtalk-expand} \\ &\equiv C^{\rm I}_{ij, i'j'} + C^{\rm X}_{ij, i'j'}
\end{align}
where for the equality \ref{fullcov-xtalk-expand} we have expanded to linear order in the attenuation parameters, and in the last line we have defined the covariance matrix of an ideal interferometer $C^{\rm I}$ and the contribution due to cross-talk $C^{\rm X}$. We can again gain some information by looking at the various possibilities for the entries of this matrix, however now we will concentrate on the cross-talk contributions. For the diagonal components we obtain
\begin{align}
(\sigma^{\rm X}_{ij})^2 = (S + N)_{i}(\widetilde{V}_{jj} + \widetilde{N}_{jj})^* + (\widetilde{V}_{ii} + \widetilde{N}_{ii})(S + N)_{j}.
\end{align}
The cross-talk contribution to the variance depends on the auto of the cross-talk contributions to the visibilities. Referring to Eq. \ref{tilde-def} these are given by
\begin{align}
\widetilde{V}_{ii} &= \sum_k \alpha_{ik} V_{ki} {\rm e}^{\omega_{ik}} + \sum_l \alpha^*_{il} V_{il} {\rm e}^{-\omega_{il}},
\\ \widetilde{N}_{ii} &= \beta_{ii} N e^{\omega_{ii}} + \beta_{ii}^*N e^{- \omega_{ii}},
\nonumber \\ &= \beta_{ii}N \left( e^{\omega_{ii}} + e^{- \omega_{ii}} \right),
\\ & = \beta_{ii}N 2i \sin{\left( 2\pi \tau_{ii} \right)}.
\end{align}
The signal contribution is complicated, but the auto term is simplified by having only one feed receiving delayed contributions instead of two. The noise contribution depends on the $\tau_ii$ parameter, which represents the posibility that a feed's own noise is delayed and correlated with the original version. The off diagonal terms become
\begin{align}
(i=i', j\neq j') &\rightarrow (S + N)_{i} (\widetilde{V}_{jj'} + \widetilde{N}_{jj'})^* +  (\widetilde{V}_{ii} + \widetilde{N}_{ii})V^*_{jj'} ,
\\{\rm and~}(i\neq i', j\neq j') &\rightarrow V_{ii'}(\widetilde{V}_{jj'} + \widetilde{N}_{jj'})^* +  (\widetilde{V}_{ii'} + \widetilde{N}_{ii'})V^*_{jj'}.
\end{align}
%The first term $\RN{1}$ contains the standard correlated contribution to the covariance, plus products of cross-talk and either signal or noise terms. Still working to linear order in the $\alpha$s
%\begin{subequations}
%\begin{align}
%\RN{1} = &~ (V_{ij} + \delta_{ij}T_{\rm sys} + \widetilde{V}_{ij} + \widetilde{N}_{ij})(V_{i'j'} + \delta_{i'j'}T_{\rm sys} + \widetilde{V}_{i'j'} + \widetilde{N}_{i'j'})^*
%\label{xtalk-expansion-1-a}  \\
% = &~  (V_{ij} + \delta_{ij}T_{\rm sys})(V_{i'j'} + \delta_{i'j'}T_{\rm sys})^*
% \label{xtalk-expansion-1-b} 
%\\ &+ V_{ij}\widetilde{V}^*_{i'j'} + \widetilde{V}_{ij}V^*_{i'j'}
%\label{xtalk-expansion-1-c} 
%\\ &+ V_{ij}\widetilde{N}^*_{i'j'} + \widetilde{N}_{ij}V^*_{i'j'}
%\label{xtalk-expansion-1-d}
%\\ &+ \delta_{ij}T_{\rm sys}\widetilde{V}^*_{i'j'} + \widetilde{V}_{ij}\delta_{i'j'}T_{\rm sys}
%\label{xtalk-expansion-1-e}
%\\ &+ \delta_{ij}T_{\rm sys}\widetilde{N}^*_{i'j'} + \widetilde{N}_{ij}\delta_{i'j'}T_{\rm sys} \label{xtalk-expansion-1-f}
%\end{align}  
%\end{subequations}
%Now taking the $i=i'$ and $j=j'$ component, the terms in \ref{xtalk-expansion-1-c} and \ref{xtalk-expansion-1-d} just multiply. However, the Kroenecker deltas in \ref{xtalk-expansion-1-e} and \ref{xtalk-expansion-1-f} pick out the diagonal components of the matrices they multiply. We thus obtain, for example for the first signal-like term,
%\begin{align}
%\delta_{ij}T_{\rm sys}\widetilde{V}^*_{i'j'}  &=  \delta_{ij}T_{\rm sys}\left(  \sum_{k'} \alpha^*_{i'k'} V^*_{k'j'} {\rm e}^{-\omega_{i'k'}} + \sum_{l'} \alpha_{j'l'} V^*_{i'l'} {\rm e}^{\omega_{j'l'}} \right)
%\nonumber \\
%&\rightarrow \delta_{ij}T_{\rm sys}\left(  \sum_{k'} \alpha^*_{ik'} V^*_{k'i} {\rm e}^{-\omega_{ik'}} + \sum_{l'} \alpha_{il'} V^*_{il'} {\rm e}^{\omega_{il'}} \right) = \delta_{ij} T_{\rm sys} \widetilde{V}^*_{ii} \equiv \delta_{ij} T_{\rm sys} \widetilde{T}_{{\rm A}, i}, \label{defAtilde}
%\end{align}
%and for the first noise-like term,
%\begin{align}
%\delta_{ij}T_{\rm sys}\widetilde{N}^*_{i'j'} \rightarrow \delta_{ij}T_{\rm sys}^2 \left( \alpha_{ii} {\rm e}^{\omega_{ii}} + \alpha_{ii}^*{\rm e}^{-\omega_{ii}}\right) \equiv \delta_{ij}T_{\rm sys} \widetilde{T}_{{\rm sys}, i}^*. \label{defNtilde}
%\end{align}
%In Eqs. \ref{defAtilde} and \ref{defNtilde}, we have defined the antenna and system temperatures due to cross-talk, $\widetilde{T}_{{\rm A}, i}$ and $\widetilde{T}_{{\rm sys}, i}$.
%We proceed similarly for the second term $\RN{2}$, which, we saw in Section \ref{basecalc}, usually contains the antenna-by-antenna contribution. The expansion is identical to Eqs. \ref{xtalk-expansion-1-b} - \ref{xtalk-expansion-1-f}, but with the indices apropriately permuted. We obtain
%\begin{subequations}
%\begin{align}
%\RN{2} = &~ (V_{ii'} + \delta_{ii'}T_{\rm sys} + \widetilde{V}_{ii'} + \widetilde{N}_{ii'})(V_{jj'} + \delta_{jj'}T_{\rm sys} + \widetilde{V}_{jj'} + \widetilde{N}_{jj'})^*
%\\ = &~  (V_{ii'} + \delta_{ii'}T_{\rm sys})(V_{jj'} + \delta_{jj'}T_{\rm sys})^*
%\\ &+ V_{ii'}\widetilde{V}^*_{jj'} + V_{ii'}\widetilde{N}^*_{jj'} + \delta_{ii'}T_{\rm sys}\widetilde{V}^*_{jj'} +  \widetilde{N}_{ii'}\delta_{jj'}T_{\rm sys} + h.c.
%\\ \rightarrow &~ (T_{\rm A} + T_{\rm sys})_i(T_{\rm A} + T_{\rm sys})_j^*
%\\ &+  T_{{\rm A}, i}\widetilde{T}_{{\rm A}, j}^*+ T_{{\rm A}, i}\widetilde{T}_{{\rm sys}, j}^* +  T_{{\rm sys}, i}\widetilde{T}_{{\rm A}, j}^* + T_{{\rm sys}, i}\widetilde{T}_{{\rm sys}, j}^* +  h.c.
%\end{align}
%\end{subequations}
%Collecting, we obtain the final results for the data covariance matrix of the visibilities in the presence of cross-talk
%\begin{align}
%C_{ij,i'j'} = &~ C^S_{ij,i'j'} + C^N_{ij,i'j'} + C^X_{ij,i'j'}
%\nonumber \\ = &~  (V_{ij} + \delta_{ij}T_{\rm sys})(V_{i'j'} + \delta_{i'j'}T_{\rm sys})^* + (V_{ii'} + \delta_{ii'}T_{\rm sys})(V_{jj'} + \delta_{jj'}T_{\rm sys})^*
%\nonumber \\ &+ V_{ij}\widetilde{V}^*_{i'j'} + V_{ij}\widetilde{N}^*_{i'j'} + \delta_{ij}T_{\rm sys}\widetilde{V}^*_{i'j'} +  \widetilde{N}_{ij}\delta_{i'j'}T_{\rm sys} + h.c.
%\nonumber \\ &+ V_{ii'}\widetilde{V}^*_{jj'} + V_{ii'}\widetilde{N}^*_{jj'} + \delta_{ii'}T_{\rm sys}\widetilde{V}^*_{jj'} +  \widetilde{N}_{ii'}\delta_{jj'}T_{\rm sys} + h.c.
%\end{align}
%and for the variance
%\begin{align}
%\sigma^2_{ij} = &~ \sigma^2_{S, ij} + \sigma^2_{N,ij} + \sigma^2_{X,ij}
%\\ = &~ (V_{ij} + \delta_{ij}T_{\rm sys})(V_{ij} + \delta_{ij}T_{\rm sys})^* + (T_{\rm A} + T_{\rm sys})_i(T_{\rm A} + T_{\rm sys})_j^*
%\\ &+ V_{ij}\widetilde{V}^*_{ij} + V_{ij}\widetilde{N}^*_{i'j'} + \delta_{ij}T_{\rm sys} \widetilde{T}_{{\rm A}, i}^* + \delta_{ij}T_{\rm sys} \widetilde{T}_{{\rm sys}, i}^* + h.c.
%\\ &+  T_{{\rm A}, i}\widetilde{T}_{{\rm A}, j}^*+ T_{{\rm A}, i}\widetilde{T}_{{\rm sys}, j}^* + T_{{\rm sys}, i}\widetilde{T}_{{\rm A}, j}^* + T_{{\rm sys}, i}\widetilde{T}_{{\rm sys}, j}^* +  h.c.
%\end{align}

%\subsection{Standard m-mode calculation}
%\label{app:hol:sec:noise:ss:fullcalc}
%In order to understand the full structure of the covariance matrix we must take into account the fact the voltage measured at some feed $i$, 
%\begin{equation}
%E_i(\nu ) =  \int d^2\hat{n}~E(\hat{n}, \nu)  {\rm e}^{2\pi i \hat{n}\cdot \vec{b}_{i}(\phi)},
%\end{equation}
%correlates with that measured at some other feed $j$, only when the same electric field signal arrives there as well
%\begin{equation}
%\left\langle E(\hat{n}, \nu) E(\hat{n}', \nu') \right\rangle \propto \delta(\hat{n} - \hat{n}') \delta(\nu - \nu') T(\hat{n}, \nu).
%\end{equation}
%This gives the result for the signal covariance that
%\begin{equation}
%\left\langle V_{ij}V^*_{i'j'} \right\rangle = B^\nu_{ij, \ell m} \left\langle a^{\nu}_{\ell m} a^{*\nu'}_{l'm'} \right\rangle B^{*\nu'}_{\ell'm', i'j'} = \delta_{mm'}\delta_{\ell \ell'} C_{\ell}(\nu, \nu'),
%\end{equation}
%The $\ell m$ structure of the covariance matrix depends on the sky statistics, but is diagonal for a Gaussian random field. 
%\subsection{$m$-mode calculation with cross-talk} 
%\label{app:hol:sec:noise:ss:fullcalcx}

%%%%%%%%%%%%%%%%%%%%%%%%%%%%%%%%%%%%%%%%%%%%%%%%%%%%%%%%%
%%%%%%%%%%%%%%%%%%%%%%%%%%%%%%%%%%%%%%%%%%%%%%%%%%%%%%%%%

\chapter{Appendices to Chapter \ref{chap:mapmaking}}

\section{Proof of equation (\ref{PRODUCT})} 
\label{app:mm:sec:proof}

We can show Eq. \eqref{PRODUCT} by direct multiplication,
\begin{align}
&\left(\mathbf{A} \otimes \mathbf{I} + \mathbf{u}\mathbf{u}^\dagger \otimes \mathbf{H} \right) 
 \left( \mathbf{A}^{-1} \otimes \mathbf{I} - \mathbf{A}^{-1}\mathbf{u}\mathbf{u}^\dagger\mathbf{A}^{-1} \otimes \left(\mathbf{I} + \mathbf{u}^\dagger\mathbf{A}^{-1}\mathbf{u} \mathbf{H}\right)^{-1} \mathbf{H} \right)
 \nonumber \\
 &= \mathbf{I} \otimes \mathbf{I} + \mathbf{u}\mathbf{u}^\dagger\mathbf{A}^{-1} \otimes \mathbf{H} - \mathbf{u}\mathbf{u}^\dagger\mathbf{A}^{-1}\otimes \left(\mathbf{I} + \mathbf{u}^\dagger\mathbf{A}^{-1}\mathbf{u} \mathbf{H}\right)^{-1} \mathbf{H}
\nonumber \\
&~- \mathbf{u}\mathbf{u}^\dagger\mathbf{A}^{-1}\mathbf{u}\mathbf{u}^\dagger\mathbf{A}^{-1}\otimes \mathbf{H} \left(\mathbf{I} + \mathbf{u}^\dagger\mathbf{A}^{-1}\mathbf{u} \mathbf{H}\right)^{-1} \mathbf{H}. \label{A1}
\end{align}
The product on the last line of Eq. \eqref{A1} has formed a scalar, which can be brought through the tensor product,
\begin{align}
&= \mathbf{I} \otimes \mathbf{I} + \mathbf{u}\mathbf{u}^\dagger\mathbf{A}^{-1} \otimes \mathbf{H} - \mathbf{u}\mathbf{u}^\dagger\mathbf{A}^{-1}\otimes \left(\mathbf{I} + \mathbf{u}^\dagger\mathbf{A}^{-1}\mathbf{u} \mathbf{H}\right)^{-1} \mathbf{H}
\nonumber \\
&~- \mathbf{u}\mathbf{u}^\dagger\mathbf{A}^{-1}\otimes \mathbf{u}^\dagger\mathbf{A}^{-1}\mathbf{u}\mathbf{H} \left(\mathbf{I} + \mathbf{u}^\dagger\mathbf{A}^{-1}\mathbf{u}\mathbf{H}\right)^{-1} \mathbf{H}. 
\end{align}
The last two terms can then be factored,
\begin{align}
=& \mathbf{I} \otimes \mathbf{I} + \mathbf{u}\mathbf{u}^\dagger\mathbf{A}^{-1} \otimes \mathbf{H} 
\\ \nonumber &- \left( \mathbf{I} \otimes \mathbf{I} + \mathbf{I} \otimes \mathbf{u}^\dagger\mathbf{A}^{-1}\mathbf{u} \mathbf{H}\right)
 \left( \mathbf{u}\mathbf{u}^\dagger\mathbf{A}^{-1} \otimes \left(\mathbf{I} + \mathbf{u}^\dagger\mathbf{A}^{-1}\mathbf{u}\mathbf{H}\right)^{-1} \mathbf{H}\right)
\\ =& \mathbf{I} \otimes \mathbf{I} + \mathbf{u}\mathbf{u}^\dagger\mathbf{A}^{-1} \otimes \mathbf{H} 
\\ \nonumber &- \left( \mathbf{I} \otimes \mathbf{I} + \mathbf{u}^\dagger\mathbf{A}^{-1}\mathbf{u} \mathbf{H} \right)
 \left( \mathbf{u}\mathbf{u}^\dagger\mathbf{A}^{-1} \otimes \left(\mathbf{I} + \mathbf{u}^\dagger\mathbf{A}^{-1}\mathbf{u}\mathbf{H}\right)^{-1} \mathbf{H}\right).
 \end{align}
 Multiplying through we obtain
 \begin{align}
 =& \mathbf{I} \otimes \mathbf{I} + \mathbf{u}\mathbf{u}^\dagger\mathbf{A}^{-1} \otimes \mathbf{H} \nonumber 
\\ &- \mathbf{u}\mathbf{u}^\dagger\mathbf{A}^{-1} \otimes \mathbf{H}.
\end{align}
The result can similarly be shown for multiplication on the left.

\section{Point source removal maps with CHIME Pathfinder data}
\label{app:mm:sec:pfdata}

Here, in Figures \ref{taua_comparison}  and \ref{casa_comparison}, we show the Cartesian projections 25 degrees around the expected locations of TauA and CasA, respectively, for the Stokes I maps from CHIME Pathfinder at the four selected frequencies 715.62, 683.98, 624.22, and 596.88 Mhz included in the analysis of Section \ref{ch:mm:sec:pfdata}.

\begin{figure}
\centering
\includegraphics[width=0.95\textwidth]{figures/pfmaps/taua_avhol_theory_rem_comp.png}
\caption{Cartesian projected Stokes I maps from CHIME Pathfinder pass1-2 night, 25 degrees around the expected location of TauA. The rows correspond to the selected frequencies at 715.62, 683.98, 624.22, and 596.88 Mhz which overlap the night stack and holography beams. The two left and two right columns show the maps made using the holographic and theoretical beams, respectively. The first column in each half shows the standard Wiener filtered map while the second shows the residuals after point source removal.}
\label{taua_comparison}
\end{figure}

\begin{figure}
\centering
\includegraphics[width=0.95\textwidth]{figures/pfmaps/casa_avhol_theory_rem_comp.png}
\caption{Cartesian projected Stokes I maps from CHIME Pathfinder pass1-2 night, 25 degrees around the expected location of CasA. The rows correspond to the selected frequencies at 715.62, 683.98, 624.22, and 596.88 Mhz which overlap the night stack and holography beams. The two left and two right columns show the maps made using the holographic and theoretical beams, respectively. The first column in each half shows the standard Wiener filtered map while the second shows the residuals after point source removal.}
\label{casa_comparison}
\end{figure}

TauA is the calibrator source, therefore the maps and removal quality are most similar between holography and theory since the visibilities are set to have amplitude 1 and phase 0 for all feeds and frequencies there. Still, we observe for the holographic removed maps that the tracks from TauA through the side-lobes are all but entirely removed.

For CasA, we see the most obvious evidence that frequency dependent effects on the diffuse component have been corrected by the averaged holographic beam in both the base map and removed case. This is clear by inspecting the background surrounding the source as a function of frequency (rows).

\begin{acknowledgements}

By obligation and by right, I must first thank my parents Monica and David. Their support throughout my schooling has been truly tireless. This applies analagously to Mata, and to my sisters Laura and Sarah, who have been with me through it all.

To my advisors, Profs. J. Richard Bond and Ue-Li Pen, I would like to thank for giving me the opportunity and freedom to pursue my unabridged academic interests, or simply, my dream. I recall a meeting with Ue-Li in my first year, where more as a question than a complaint I stuttered: ``I'm trying learn both cosmology \textit{and} radio astronomy," to which he could only smile and nod approvingly. In similar moments of doubt at the keyboard, I'm often bolstered by the memory of Dick's refusal to allow me to ``give up on the Grail,'' some days before my poster submission was due. Thanks are due, as well, to my Committee members Kendrick Smith and Keith Vanderlinde, for their direction throughout these past four years. Also to Laura Newburgh and Niels Oppermann for helping me find my self-confidence. And to Marcelo Alvarez for teaching me to debug.

There are a large number of names further upon who's work this thesis rests and who have served as mentors, motivators, and role models to me throughout the process. I must thank the entire CHIME collaboration. I'll single out Richard Shaw and Kiyo Masui, on whose code mine rests and is a lazy forgery, and Seth Siegel, who produced the pass1-2 night stack from CHIME Pathfinder data used in Chapter \ref{chap:mapmaking}. Furthermore, many thanks to the DRAO staff, including Tom Landecker, Tim Robishaw, Kory Phillips, and Ev Sheehan. Last but not least to the CITA computing staff; John Dubinsky and Claire Yu, and the Scinet computing staff, I would not be here now without your help and support.

\end{acknowledgements}
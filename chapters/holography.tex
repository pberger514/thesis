\chapter{\label{chap:hol} Holography}

Test.
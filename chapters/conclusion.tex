\chapter{Conclusion}

\section{SPIE Paper Conclusion} \label{ch:conclusion:sec:hol}

In this chapter, we have described and validated our technique of radio holography of bright astronomical point sources for obtaining high signal-to-noise (S/N) measurements, with good angular resolution, of the two-dimensional primary beams of the CHIME Pathfinder array across its frequency band. We have reported our progress in equipping the John A. Galt 26 m telescope with custom instrumentation for the purpose, and displayed the output of the method for a preliminary data set of 8 sources. We have described the development of a data analysis pipeline which maps these measurements exactly onto the sky, converges to the true beam in the limit of dense declination sampling, and produces beam transfer matrices which are the direct input into the Pathfinder mapmaking software. The recovered beams can then be validated based on their ability to reduce artefacts and correct beam-induced errors seen in maps made with minimal beam information. This discussion is the topic of Chapter \ref{chap:mapmaking}.

In addition to the holography method presented here, CHIME is pursing other methods of filling in the NS beam structure, notably with satellites\citep{hol2, sat2}, drones \citep{drone}, and pulsar holography. In its current form the data presented here serve as a basis for an understanding of the primary beam of a realistic cylindrical telescope array, such as CHIME and its Pathfinder.

In this document, we have described and validated our technique of radio holography of bright astronomical point sources for obtaining high signal-to-noise (S/N) measurements, with good angular resolution, of the two-dimensional primary beams of the CHIME Pathfinder array across its frequency band. We have reported our progress in equipping the John A. Galt 26 m telescope with custom instrumentation for the purpose, and displayed the output of the method for a preliminary data set of 7 sources of minimal depth. 

It is clear from the data that more integration time is necessary for the low S/N sources to begin to measure the sidelobe structure seen in the best sources. From Figure \ref{full2dbeam} we see that the seven sources for which we have holography do not fully sample the two-dimensional structure in our model of the CHIME beams. The basic program is to use these holographic measurements to test and refine our beam models. Additionally, we plan to augment our holographic observations with additional sources which were not favorably located during the period these data were collected. However, our analysis of Section \ref{ch:hol:sec:sims} suggests that there is a minimum primary source flux at which one can expect to obtain a reasonable beam trace. In addition to the holography method presented here, we are pursing other methods of filling in the NS beam structure, notably with satellites\citep{hol2, sat2}, drones \citep{drone}, and pulsar holography. In its current form the data presented here serve as a basis for an understanding of the primary beam of a realistic cylindrical telescope array, such as CHIME and its Pathfinder.

\section{Mapmaking Chapter Conclusion}
\label{ch:conclusion:sec:mmodes}

In this chapter, we have shown that the $m$-mode formalism of \cite{mmodes1, mmodes2} is indeed convenient for analysis of full-sky data from transit interferometers, even when the assumption of statistical isotropy of the sky is relaxed. In Section \ref{ch:mm:sec:model} we adopted a realistic sky model that includes components, namely (although not limited to) bright point sources, which display the relationship between statistical anisotropy and a covariance matrix that is non-diagonal in $m$. Furthermore, in Section \ref{ch:mm:sec:sims}, we demonstrated with simulations how such components cause heavily non-local ringing in a standard attempt to optimally estimate a map. We then showed how one can use a linear, optimal Wiener filter reconstruction technique to project out the components in the same step as the deconvolution and map making. For the task of inverting the covariance matrix, we developed an algorithm based on the Sherman-Morrison-Woodbury formula (Section \ref{ch:mm:sec:algorithm}), which adds minimal computational cost to a method that assumes a block diagonal in $m$ structure. Indeed, as mentioned at the end of Section \ref{ch:mm:sec:model}, this algorithm can be used to efficiently either estimate or project-out any low-rank component for which the diagonalization transformation is known. In Section \ref{ch:mm:sec:sims}, we described the simulation technique used to validate our method and computed the spectrum of residuals, which were shown to be small compared to the estimated signal. In Section \ref{ch:mm:sec:freq}, we discussed the computational difficulty of considering the full frequency structure of the data covariance matrix, but suggested several viable solutions.

Furthermore, in Section \ref{ch:mm:sec:pfdata}, we applied the method to CHIME Pathfinder data, specifically the pass1-2 night stack which was collected over a $\sim$8 month observing period in late 2015 and 2016. We applied the tools developped in the previous sections to analyze the quality of the deconvolution provided by the holographic beams constructed in Chapter \ref{chap:hol}. By comparison of the Stokes I maps before and after point source removal, we found that the (averaged) holographic beams provide a minimum factor of 2 improvement in the complex response at the locations of the measured point sources. We further found a large body of qualitative evidence that the inclusion of the correct level of side lobe structure corrects frequency dependent artefacts in the diffuse component.

Further work will develop better methods for estimating the covariance of the visibilities (Appendix \ref{app:hol:sec:noise}) and tune the mapmaker to become an accurate estimator of point source photometry. This will pave the way for an analysis of holographic beams averaged by baseline, the exact solution for a stack averaged across redundant baselines (at that point this is more of a computational than technical task, and the same analysis presented here can be repeated).

As a whole, this chapter demonstrates the usefulness of the $m$-mode analysis method, for example, for efforts to map the synchrotron emission of the Milky Way \cite{eastwoodetal} or the cosmic 21-cm intensity, on the full-sky.
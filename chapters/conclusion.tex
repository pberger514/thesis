\chapter{Conclusion}

\section{SPIE Paper Conclusion} \label{ch:conclusion:sec:hol}

In this document, we have described and validated our technique of radio holography of bright astronomical point sources for obtaining high signal-to-noise (S/N) measurements, with good angular resolution, of the two-dimensional primary beams of the CHIME Pathfinder array across its frequency band. We have reported our progress in equipping the John A. Galt 26 m telescope with custom instrumentation for the purpose, and displayed the output of the method for a preliminary data set of 7 sources of minimal depth. 

It is clear from the data that more integration time is necessary for the low S/N sources to begin to measure the sidelobe structure seen in the best sources. From Figure \ref{full2dbeam} we see that the seven sources for which we have holography do not fully sample the two-dimensional structure in our model of the CHIME beams. The basic program is to use these holographic measurements to test and refine our beam models. Additionally, we plan to augment our holographic observations with additional sources which were not favorably located during the period these data were collected. However, our analysis of Section \ref{ch:hol:sec:sims} suggests that there is a minimum primary source flux at which one can expect to obtain a reasonable beam trace. In addition to the holography method presented here, we are pursing other methods of filling in the NS beam structure, notably with satellites\citep{hol2, sat2}, drones \citep{drone}, and pulsar holography. In its current form the data presented here serve as a basis for an understanding of the primary beam of a realistic cylindrical telescope array, such as CHIME and its Pathfinder.
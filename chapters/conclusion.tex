\chapter{Conclusion and Outlook}
\label{chap:conclusion}

In the previous three chapters, we have presented developments in several sections along the flow of the analysis pipeline.

In Chapter \ref{chap:signal}, we developped a simulation and analysis framework for fast and accurate modeling of cosmic neutral hydrogen based on the Peak Patch framework. Our modification of the Peak Patch code permitting generation of coherent initial conditions on the fly is a novel, powerful, and flexible application of multi-scale inital conditions generation. It permitted us to generate high resolution simulations in a cosmological volume (Section \ref{ch:signal:sec:sims}), allowing us resolve up to a few percent the entire population of halos contributing to the 21 cm line intensity mapping signal. The analysis of Lagrangian biasing of this population illuminated a novel subgrid field biasing scheme. When combined with the standard Peak Patch code this provides a mock intensity map generation method with unprecedented speed and accuracy.

In Chapter \ref{chap:hol} we described and validated our technique of radio holography of bright astronomical point sources for obtaining high signal-to-noise (S/N) measurements, of the two-dimensional primary beams of the CHIME Pathfinder array across its frequency band. We have equipped the John A. Galt 26 m telescope with custom instrumentation for the purpose, and have undertaken a campaign of holography observations. We then described the development of a data analysis pipeline which coverts the time ordered data to beam transfer matrices, the format which can directly be input into the CHIME Wiener filtered mapmaker. This method maps the measured holographic information exactly onto the sky and converges to the true beam in the limit of dense declination sampling.

In Chapter \ref{chap:mapmaking}, we have shown that the $m$-mode formalism of \cite{mmodes1, mmodes2} is indeed convenient for analysis of full-sky data from transit interferometers, even when the assumption of statistical isotropy of the sky is relaxed. We then showed how one can use a linear, optimal Wiener filter reconstruction technique to project out non-stationary components in the same step as the deconvolution and mapmaking. Furthermore, in Section \ref{ch:mm:sec:pfdata}, we applied the method to CHIME Pathfinder data, specifically the pass1-2 night stack which was collected over a $\sim$8 month observing period in late 2015 and 2016. We applied the tools $m$-mode formalism to asses the quality of the deconvolution provided by the holographic beams constructed in Chapter \ref{chap:hol}. By comparison of the Stokes I maps before and after point source removal, we found that the (averaged) holographic beams provide a minimum factor of 2 improvement over the fiducial theoretical model.

In addition to the holography method presented here, CHIME is pursing other methods of filling in the NS beam structure, notably with satellites\citep{hol2, sat2}, drones \citep{drone}, and pulsar holography. These methods could combine to create a completely empirical beam model or alternatively the considerations of Appendix \ref{app:hol:sec:model} could be used to map the multi-declination measurements to cylinder distortions. Indeed the latter application is the etymology of the term radio ``holography''. In feedback with electromagnetic simulations such measurements could develop a first principles understanding of the scattering and coupling effects governing the beam the shape (see Section \ref{ch:hol:sec:analysis}). As well, we have use only the Stokes I information present in the holography data and adopted a simple model of polarisation. Although similar methods could be developped to treat each Stokes parameter independently, this over-complicates the problem as there are many less degrees of freedom in the polarized beam\citep{holpol}. In any case, the simplicity, directness, and exactness of the current method will provide a useful comparison for any more sophisticated treatment.

Furthermore, we have resticted the analysis to map-based analyses of the simulations and data presented in Chapter \ref{chap:mapmaking}, where effects can most easily be interpreted in terms of errors in the beam model. However, in targeting the 21 cm line emission from neutral hydrogen, the signal component is statistically isotropic and lives naturally in Fourier space \citep{mmodes2}. In combination with the signal component developped in Chapter \ref{chap:signal} and the foreground simulation methods of Chapter \ref{chap:mapmaking}, power spectrum analysis provides a powerful method for assessing the accuracy of the beams \citep{wedge1, wedge2}. In addition, the flexibility, speed, and accuracy of the subgrid-augmented Peak Patch method is an ideal tool for exploring the more sophisticated methods discussed in Chapter \ref{chap:intro}, in the full context of instrumental and astrophysical effects.

We see, therefore, that the methods developped in this thesis, while strongly rooted in the reality of day-to-day observation of a state-of-the-art 21 cm experiment such as CHIME and its Pathfinder, are well-suited to achieve target sensitivity and perform end-game, optimal analyses. Through such a combination of simulation and forward-modeling in feedback with direct analysis of the data, the field of 21 cm line intensity mapping can progress past the current hurdles of a detection of the auto-power spectrum and Baryon Acoustic Oscillations, to a strongly competitive method for mapping and interpreting the observable Universe.
\begin{abstract}
% (At most 150 words for M.Sc. or 350 words for Ph.D.)
The technique of 21 cm line intensity mapping (LIM) has emerged as a promising new frontier in the field of observational cosmology, promising fast tomographic mapping of enormous volumes at the angular resolution needed to resolve Universe's large-scale structure. The Canadian Hydrogen Intensity Mapping Experiment (CHIME) is a new cylindrical transit interferometer, begun observations at the Dominion Radio Astrophysical Observatory in Penticton, British Columbia. A two cylinder test-bed -- the CHIME Pathfinder -- has been surveying the Northern hemisphere in 1024 frequency bands between 400 and 800 MHz since the fall of 2015. These telescopes are optimized for 21 cm LIM at redshifts $0.8-2.5$, targeting a time-resolved detection of the Baryon Acoustic Oscillations. Control of the systematics accompanied by modern interferometers has instituted a new philosophy of ``end-to-end modelling'' in the field: Whereby the underlying signal, astrophysical foregrounds, telescope, and receiving electronics are modelled together in feedback with the data. In this thesis we describe the development of end-to-end analysis and forward modelling techniques for 21 cm LIM. First, we describe a novel subgrid biasing scheme for dark matter halo catalogs and its application to accurate more accurate modeling of the intensity mapping signal component. The systematic focused on in this thesis is the angular response function, or ``beam,'' of the interferometer. We go on to describe the radio holography technique implemented to measure this function for each of the CHIME Pathfinder's 256 inputs and develop a data analysis pipeline capable of converting time ordered data to a beam model for input to the mapmaker. Finally, we present a modification of the $m$-mode formalism mapmaker, designed explicitly for full-sky 21 cm LIM, which relaxes the usual assumption of statistical isotropy of the signal, allowing it to treat arbitrarily realistic sky models. We then exploit these tools to deconvolve CHIME Pathfinder maps with the beam model constructed from holography, and find a minimum factor of 2 improvement over previous methods. In its entirety, this thesis forms a foundation for the methods necessary extract accurate cosmological information from modern interferometers such as CHIME and its Pathfinder.
\end{abstract}